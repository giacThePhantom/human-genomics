\chapter{Tumour heterogeneity and resistance to cancer therapies}

\section{Abstract}
As a result of cancer heterogeneity, the bulk tumour might include a diverse collection of cells harbouring distinct molecular signatures with differential levels of sensitivity to treatment.
This might reult in a non-uniform distribution of distinct subpopulations across and within disease sites or temporal variations.
This provides the fuel for resistance.

	\subsection{Introduction}
	The stochastic nature of cancer initiation reinforces the notion that the development and progression of cancer does not follow a fixed course.
	The ongoing evolution of cancer migh generate a molecularly heterogeneous bulk tumour consisting of cancer cells harbouring distinct molecular signatures with differential levels of sensitivity.
	Intertumoural heterogeneity is the heterogeneity between patients harbouring tumours of the same histological type.
	Intratumoral heterogeneity is spatial or temporal heterogeneity: dynamic variations in the genetic diversity of an individual tumour over time.
	Oncogenic drivers can be exploited to treat cancer, but almost all of them develop resistance to targeted therapies.
	Intratumoural heterogeneity drives the evolution of cancers and fosters drug resistance.
	A comprehensive understanding of tumour dynamics is essential for the development of effective and durable therapeutic strategies.

\section{Causes of intratumoral heterogeneity}

	\subsection{Genomic instability}
