\graphicspath{{chapters/papers/03/images}}
\chapter{Tumour heterogeneity and resistance to cancer therapies}

\section{Abstract}
As a result of cancer heterogeneity, the bulk tumour might include a diverse collection of cells harbouring distinct molecular signatures with differential levels of sensitivity to treatment.
This might result in a non-uniform distribution of distinct subpopulations across and within disease sites or temporal variations.
This provides the fuel for resistance.

	\subsection{Introduction}
	The stochastic nature of cancer initiation reinforces the notion that the development and progression of cancer does not follow a fixed course.
	The ongoing evolution of cancer might generate a molecularly heterogeneous bulk tumour consisting of cancer cells harbouring distinct molecular signatures with differential levels of sensitivity.
	Intertumoural heterogeneity is the heterogeneity between patients harbouring tumours of the same histological type.
	Intratumoral heterogeneity is spatial or temporal heterogeneity: dynamic variations in the genetic diversity of an individual tumour over time.
	Oncogenic drivers can be exploited to treat cancer, but almost all of them develop resistance to targeted therapies.
	Intratumoural heterogeneity drives the evolution of cancers and fosters drug resistance.
	A comprehensive understanding of tumour dynamics is essential for the development of effective and durable therapeutic strategies.

\section{Causes of intratumoral heterogeneity}

	\subsection{Genomic instability}
	Instability might result from exposure to exogenous mutagens and aberrations in endogenous processes.
	Characteristic genetic signatures associated with some of these mutagenic processes have been identified by large-scale genomic sequencing.
	Exposure to chemotherapy might increase the mutational spectrum of a tumour and create genomic instability.
	Genomic instability can also result from chromosome-level changes that lead to gains or losses of whole-genome segments rather than point mutations.

	\subsection{The clonal evolution and selection hypothesis}
	Genomic instability fosters genetic diversity by providing the raw material needed for the generation of tumour heterogeneity.
	Dynamic chromosomal instability can lead to copy-number imbalances and non-uniform loss of chromosomal segments harbouring specific alterations that can contribute to mutational heterogeneity across different regions.
	Increased levels of genomic instability promote the emergence of more competitive subclones.
	Genomic instability cooperates with other factors to promote the development of tumour heterogeneity.
	The clonal evolution method and or the selection framework are used to explain how clonal diversity is generated and maintained.
	This model is based on the hypothesis that tumour initiation occurs in a stochastic manner, beginning with an induced change that confers a selective growth advantage and leads to neoplastic proliferation.
	The genomic instability creates additional genetic diversity subjected to evolutionary selection pressures, resulting in the sequential emergence of increasingly genetically abnormal and heterogeneous subpopulations.
	Linear evolution describes evolution owing to the successive acquisition of mutations that confer a growth and survival advantage.
	Sequential clones have advantageous mutations and out-compete ancestral clones.
	Alternatively branching evolution denotes the emergence and divergent propagation of multiple sub clonal tumour cell populations that share a common ancestor.
	Branched evolution has a greater opportunity to create a more heterogeneous tumour.
	Moreover different sub clones might cooperate for tumour propagation in cancer.
	This is shown by polyclonally seeded metastatic sites.

\section{The spectrum of tumour heterogeneity}

	\subsection{Spatial heterogeneity}
	Cancer can ignore growth suppression signals, invade local tissues and metastasize to distant organs.
	The molecular make-up of cancer cells in different sites can be different, owing to the variable influences of micro-environment related factor.
	Heterogeneity might exist among the cell present within the parent tumour.
	The uneven distribution of diverse tumour subpopulations across different sites and within a tumours is termed spatial heterogeneity.

		\subsubsection{Heterogeneity at a single disease site}
		Primary tumours contain multiple geographically separated and molecularly distinct cellular subpopulations.
		This can result in an uneven distribution of key molecular alterations across different regions.
		It might manifest as the ubiquitous presence of key molecular driver alterations, with an unequal distribution of additional molecular alterations.
		The pattern of spatial heterogeneity observed is reflective of the specific evolutionary context.
		Multiregion sampling is an informative investigational strategy that improves the ability to determine the extent of spatial heterogeneity within an individual tumour.
		Many of the unevenly distributed passenger mutations are not expressed.
		Markers of different impact can be present in geographically distinct regions within the same tumour.
		Genomic instability is a better biomarker than the alterations detected.
		A substantial level of genetic diversity exists between individual cancer cells.
		Multifocal tumours (multiple histologically similar cancers within a single organ) pose a unique challenge because genetic homogeneity cannot be assumed.
		Moreover the potential exists for divergence.

		\subsubsection{Comparison of spatially distinct disease sites}
		The genetic make-up of cancer cells at a specific metastatic site might differ from that of the parent tumour.
		The degree of genetic discordance might reflect whether the metastases occurred as late events or arose through dissemination early (less discordant) in the course of tumour development.
		Comparison of the genetic make-up of different metastases reveal substantial levels of heterogeneity.
		In the simplest scenario, seeding of multiple metastatic sites by identical clones, all metastatic sites would have the same genetic signature.
		This uni-directional flow might not be a universal scenario: tumour self-seeding and exchange of tumour material between different metastatic sites can occur.
		Moreover polyclonal seeding can happen.
		In some cases distant metastases can arise from independent seeding by genetically distinct subclones originating from the primary tumour.
		Moreover site specific factors could promote genetic divergence after initial colonization.

	\subsection{Temporal heterogeneity}
	Temporal heterogeneity refers to the dynamic variation in the genetic diversity of a tumour over time.
	Chemotherapy can alter the molecular make-up of tumours over time by creating shifts in the mutational spectrum.
	Mutations in genes that are fundamental to replication and cell-cycle regulation can contribute to genomic instability.
	Targeted therapies can extort selective pressures on oncogene-driven cancer cells.

		\subsubsection{Genomic complexity might increase with exposure to targeted therapies}
		The efficacy of targeted therapies reflects therapeutic vulnerabilities resulting from a dependence on specific growth signals and the original location of the driver alteration.
		Resistance can arise through mutations, activation of bypass signalling pathways and cell-lineage changes.
		De novo resistance alterations can be present at low variant allele frequencies in pretreatment tumour specimens.
		Resistant clones emerge from the selective expansion of pre-existing populations during treatment with targeted agents.
		The genomic complexity increases with exposure to sequencing systemic therapies: the single genetic snapshot depicted in a diagnostic biopsy sample might become outdated during the clinical course.
		Serial characterization of tumours at multiple time points is necessary in order to accurately capture the various temporal shifts that take place during clonal evolution.

		\subsubsection{Longitudinal sampling provides insight into temporal heterogeneity}
		Longitudinal profiling has the potential to decipher the role of clonal evolution.
		Repeat biopsy sampling enables the tailored use of sequential therapies.
		Clonal evolution that arises from the selective pressures created by targeted agents is dynamic.
		Clonal dynamics are not always easily manipulated by treatment interruption.
		Longitudinal sampling might be most clinically relevant when used as a tool to enable the selection of subsequent treatment strategies.

		\subsubsection{Residual drug-tolerant cells can foster temporal heterogeneity}
		A reliance on biopsy samples might fail to detect cancers at the early or intermediate stages of resistance.
		The residual disease left to therapies could harbour a small population of quiescent drug-tolerant cells that have survived owing to adaptive activation.
		Acquired resistance is attributed to selective expansion of pre-existing subclonal population.
		Data from some studies suggest that the ongoing evolution of drug-tolerant cells leads to de novo generation of resistance alterations.
		These can emerge from single-cells clones derived from drug-tolerant cells.
		This emphasizes the necessity of developing sensitive technologies that enable the early detection of resistance.
		The emergence of resistance highlights the need to develop therapeutic strategies that target the minimal residual disease state.

\section{Non-invasive monitoring of heterogeneity}
Analysis involving single-site biopsy sampling might result in underestimation of the degree of spatial heterogeneity, and sampling intervals tolerable by the patient might not enable the true extent of temporal heterogeneity to be captured.
Liquid biopsies that facilitate longitudinal analysis of tumour-derived genetic material are a promising strategy for addressing the shortcomings of tissue sampling.
Genotyping of circulating tumour cells, circulating exosomes and circulating cell-free tumour DNA or ctDNA had promising results.

	\subsection{Analysis of ctDNA}
	Analysis of ctDNA is a sensitive and highly informative method of identifying clinically relevant genomic alterations with a high degree of concordance with tissue biopsy.
	ctDNA might enable the identification of alterations not detected by tissue genotyping.
	Optimizing ctDNA platforms to increase sensitivity for very-low-frequency mutations might enable the early detection of resistance and relapse.
	This is because the detection of variants associated with treatment resistance in plasma can precede the emergence of evidence of radiographic progression by $10$ months in some patients.
	Longitudinal plasma analysis is an effective tool for gauging the influence of treatment on the molecular and genetic make-up of a patient's cancer over time.
	Plasma clearance might be predictive of a clinical response.
	Serial plasma analysis can enable the kinetics of dominant alterations present before treatment to be monitored and to capture clonal shifts occurring during therapy.
	Several studies found that genotyping of plasma samples enables the kinetics of intratumoural heterogeneity to be captured in a time-frame that is potentially conducive to guiding clinical decision making.
	Plasma samples contain ctDNA from multiple metastatic sites so it can enable the detection of clinically relevant alterations that are not identified through analysis of tissue biopsy samples.
	Reliance on tissue sampling alone often underestimates the degree of overlap between distinct driver alterations.
	Heterogeneity of alterations associated with resistance in plasma samples correlates with shorter PFS (progression free interval) durations.
	Analysis of pretreatment plasma samples can provide some insight into the probable disease outcomes of patients receiving treatment.
	Sampling of multiple lesions during autopsy can improve upon the ability of tissue sampling to capture the extent of molecular heterogeneity present in cancers.
	Plasma genotyping might provide a more-comprehensive readout of tumour heterogeneity.

\section{Overcoming heterogeneity}
Higher level of intratumoural heterogeneity predispose patients to inferior responses to anticancer therapies, including to targeted agents.
The degree to which subpopulations coexist will affect clinical outcomes.
Cancers become more heterogeneous and complex with successive exposure to systemic agents: responses to subsequent lines of therapy are often not as robust as responses to initial treatments.
The current paradigm of sequential treatment is suboptimal: it fails to address the heterogeneity that might underlie incomplete responses to treatment.
A bulk solid tumour is a heterogeneous entity that predominantly consists of drug-sensitive cells: mathematical modelling might enable the design of dosing schedules that account for this inherent heterogeneity.
Withdrawal of targeted therapy can negate the selective advantage conferred upon drug-resistant cells and enable repopulation of the tumour with drug-sensitive cells.
Intermitting dose scheduling can temporarily suppress clonal outgrowth, although the effect is to subdue heterogeneity rather than to eliminate it.
Combination approaches that target heterogeneous tumour populations have proven successful in preclinical studies.
Plasticity between different signalling pathways is a potential manifestation of temporal heterogeneity under therapeutic selective pressure.
Drug combinations targeting multiple signalling pathways could provide another mean of addressing intratumoural heterogeneity.
The characteristics of the tumour before treatment could be used to design therapeutic approaches.
Drug tolerant cells can develop a wide range of resistance mechanisms and targeting this population offers another opportunity to curtail intratumour heterogeneity.
Combinations are likely to be more effective than monotherapy.
Many tumour lack actionable genetic alterations.
In these cases strategies that target more ubiquitous sources of heterogeneity are likely to be most applicable.
Genomic instability is a pervasive and ideal target.
Countering the development of genomic instability is a more daunting task than silencing a dominant signalling pathway.
This is likely to be most effective in patients with cancers that are prone to mutagenic stress.
