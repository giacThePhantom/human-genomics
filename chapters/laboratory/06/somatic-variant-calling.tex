\graphicspath{{chapters/laboratory/06/images}}
\chapter{Somatic variant calling}

\section{Introduction}

	\subsection{Genomic alterations}
	Cancer is driven by genomic alterations, among which:

	\begin{multicols}{2}
		\begin{itemize}
			\item Single nucleotide aberrations.
			\item Short insertions or deletions (indels).
			\item Copy number variations CNVs.
			\item Larger structural variations SVs.
		\end{itemize}
	\end{multicols}

	Somatic variant calling consider single nucleotide aberrations, which can be distinguished in:

	\begin{multicols}{2}
		\begin{itemize}
			\item Single nucleotide polymorphisms SNPs: mutations shared amongst a population.
				These are aberrations expected at a particular position for any member in the species.
				The are well-characterized and catalogued in dbSNP.
			\item Single nucleotide variations SNVs: private mutations.
				They occur at low frequency and are not common.
				They typically are non-synonymous mutations that result in amino acid change, impacting protein sequence and function.
				They are somatic mutations and typically tumour-specific.
		\end{itemize}
	\end{multicols}

	\subsection{Methods}
	There are more than $15$ algorithms that perform somatic variant calling, which could be divided into three categories:

	\begin{multicols}{2}
		\begin{itemize}
			\item Allele counting.
			\item Probabilistic methods that use a Bayesian model to quantify statistical uncertainty.
				They assign priors based on observed allele frequency of multiple samples to compute it.
			\item Heuristic approaches based on thresholds for read depth, base quality, variant allele frequency and statistical significance.
		\end{itemize}
	\end{multicols}

	\subsection{Choosing a caller}
	Substantial discrepancies exist among the calls from different callers.
	Callers appear to be less concordant for calling somatic SNVs than germline SNPs.
	Sensitivity and specificity vary across callers and along the genome within any caller.
	This depends on factors like:

	\begin{multicols}{2}
		\begin{itemize}
			\item Depth of sequence coverage in tumour and matched normal.
			\item The local sequencing error rate.
			\item The allelic fraction of the mutation.
			\item The evidence thresholds used to declare a mutation.
		\end{itemize}
	\end{multicols}

	MuTect claims to be more sensitive than other methods for low-allelic-fraction and low read support events while remaining highly specific.
	Multiple variant callers are needed in a pipeline to reduce false negatives.

	\subsection{JointSNVMix and VarScan2}
	JointSNVMix and VarScan2 implement the Fisher's exact test.
	The count allele data from normal and tumour sample and compare them using a two-tailed Fisher's exact test.
	If the counts are significantly different the position is labelled as a variant position $p$-value $< 0.001$.
	It the two-tailed for the Fisher's exact test $p$-value $<0.0001$ the association between groups and outcomes is considered to be extremely statistically significant.

\section{Varscan 2}

	\subsection{Introduction}
	VarScan2 calls somatic variants, SNPs and indels using an heuristic method and a statistical test based on the number of aligned reads supporting each allele.

	\subsection{Calling variants}

		\subsubsection{Tumour matches normal}
		If tumour matches normal:

		\begin{multicols}{2}
			\begin{itemize}
				\item If tumour an normal match the reference: call reference.
				\item If tumour and normal do not match the reference: all germline.
			\end{itemize}
		\end{multicols}

		\subsubsection{Tumour does not match normal}
		If tumour does not match normal the significance of allele frequency difference is computed by Fisher's exact test.
		Then:

		\begin{multicols}{2}
			\begin{itemize}
				\item If difference is significant ($p$-$value < threshold$):

					\begin{itemize}
						\item If normal matches reference: call somatic.
						\item If normal is heterozygous: call loss of heterozygosity.
						\item If tumour and normal are variant but different: call unknown.
					\end{itemize}

				\item If difference is not significant: tumour and normal reads counts are combined for each allele, the $p$-$value$ is recalculated and germline is called.
			\end{itemize}
		\end{multicols}

	\subsection{Filtering}
	Pre-processing in the mapping phase and SNV filtering help to minimize false positives:

	\begin{multicols}{2}
		\begin{itemize}
			\item Absent in dbSNP.
			\item Exclude LOH events.
			\item Retain non-synonymous coding SNVs.
			\item TUmour total reads and variant reads $\ge 3$.
			\item Mapping quality $\ge 40$ and SNV quality $\ge 20$.
			\item Max SNV calls $<3$ within a given window of $10bp$ around the site.
			\item SNV farther than a given distance $10bp$ from a predicted indel of a certain quality $\ge 50$.
			\item Stand balance or bias.
			\item Concordance across various SNV callers.
		\end{itemize}
	\end{multicols}
