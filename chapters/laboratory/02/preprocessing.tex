\chapter{Data pre-processing}

\section{Realignment}

	\subsection{Introduction}
	The identification of indels is not easy to be done by mappers.
	In particular indels across the ends or in complex regions of reads are of difficult detection.
	This generates some artefact mismatches, which if not corrected can introduce biases that propagates across the downstream analysis.
	Realignment is an operation performed on BAM file in order to improve accuracy of other pre-processing steps.

	\subsection{An example}
	A typical situation in which realignment helps is one in which a small homopolymeric region is in the middle of two regions with consecutive SNPS.
	Given the complexity the aligner is not able to identify an indel in that particolar region.
	Because of this some artefacts will be found at the end of the reads.
	The scoring function allow to accept an alignment even in the presence of gaps and mismatches.
	In this case a misalignment could introduce mismatches, reducing the score of the alignment.

	\subsection{Objective of realignment}
	The objective of realignment is to identify regions hiding indels.
	This process can be done in two steps:

	\begin{multicols}{2}
		\begin{enumerate}
			\item Known sites collected in databases like $1000$ genomes and dbSNP are checked.
			\item The CIGAR line gives information about the goodness of an alignment.
				All reads are explored and regions where indels are present should be identified.
				If this is not feasible the density of variation of quality at difference reads should be explored.
				For example an high density of mismatches with respect to expectancy suggests the presence of an indel.
		\end{enumerate}
	\end{multicols}

	\subsection{GATK}
	GATK implements realignment considering the default CIGAR line and the density of variation to provide known indes.
	To do so GATK considers each alignment and:

	\begin{multicols}{2}
		\begin{enumerate}
			\item Inserts an indel and finds a better alignment with an alternative consensus sequence.
			\item The score for that alternative consensus is computed as the total sum of the quality scores of mismatching bases.
			\item If the score of the best alternative consensus is deemed significantly better by a LOD score the proposed alignment of the reads is accepted.
		\end{enumerate}
	\end{multicols}

	\subsection{Protocol}

	\begin{multicols}{2}
		\begin{itemize}
			\item Apply RealignedTargetCreator to the BAM file to identify which regions need to be realigned.
				By default the tool uses the density criteria and the CIGAR one, but it is also possible to provide a file with a list of known sites.
			\item IndelRealigner performs the actual realignment at the RTC target intervals using the same input files.
		\end{itemize}
	\end{multicols}

	\subsection{Realignment results}
	When the new BAM file is created realigned reads change their CIGAR but maintain the original one with an OC tag.
	Because of this it is easy to check how the realignment was performed.
	Realignment is a useful step as position artefacts might lead to an incorrect correlation between a patient and pathogenic SNPs.

\section{Recalibration}

	\subsection{Introduction}
	Base quality score recalibration involves assigning accurate confidence scores to each sequence.
	Quality scores are critical for all the downstream analysis and systematic biases are a major contributor to bad calls.

	\subsubsection{Dealing with systematic errors}
	Systematic errors correlate with base call feature like:

	\begin{multicols}{2}
		\begin{itemize}
			\item Reported quality score.
			\item Position within the read due to the machine cycle.
			\item Sequence context due to the chemistry of sequencing.
		\end{itemize}
	\end{multicols}

	The error distribution and how error varies with base call features can be computed empirically.
	This process is made possible by looking at works per read groups, or entire lane of data.

	\subsection{Computing empirical qualities}
	Except for known variants any sequence mismatch represents an error.
	The number of observation and the number of errors are taken into account as a function of the various covariates that give rise to systematic errors.

		\subsubsection{Base quality score recalibration}
		A PHRED scaled quality score is computed as:

		$$\frac{\# of\ reference\ mismatches+1}{\# of\ observed\ bases +2}$$

		Base quality score recalibration or BQRS is a method that adjust the PHRED quality scores to be more accurate by looking at every base in a BAM file.
		To run BQSR an additional file is necessary to make the process aware of all known single nucleic polimoprhisms SNPs.
		Each read base is grouped into separate group bins, as different sequencing machines may be calibrated differently.
		Then the data is split in a read group by the quality scores.
		Then BQSR compare the scores reported by the sequencing machines with the empirical scores derived from the empirical error counts.
		An empirical PHRED score is computed as:

		$$Q_{actual} = Q_{global} + \Delta all + \Delta readgroup + \Delta quality + \sum\limits_i^n \Delta covariate_i$$

		Post-recalibration quality scores should fit the empirically-driven quality scores well, without obvious systematic biases.

	\subsection{Protocol}
	A typical process for base quality score recalibration is composed of $4$ steps:

	\begin{multicols}{2}
		\begin{enumerate}
			\item BaseRecalibrator: model the error modes and recalibrate qualities.
				Its inputs are a BAM file and the known sites.
			\item PrintReads: write recalibrated data to a BAM file thanks to the recalibration table produced in the previous step.
				Original qualities are retained with the OC flag.
			\item The process is repeated to build the after model to evaluate remaining error.
			\item AnalyzeCovariates: before and after plots are made based on recalibration tables.
		\end{enumerate}
	\end{multicols}

\section{Marking duplicates}

	\subsection{Introduction}
	Duplicates are non-independent measurements of a sequence, as they are sampled from the exact same template of DNA, violating the assumptions of variant calling.
	Errors in sample or library preparation will be propagated to all the duplicates.
	Therefore, the best copy among all the duplicates should be taken to mitigate the effects of errors.

	\subsection{Identification of duplicates}
	Duplicates come from the same input DNA template, so they should have the same start position on reference.
	This is even more true for paired end, in which both the reverse and the forward read should have the same starting position.
	Duplicate sets are first identified, then the representative read based on base quality scores an other criteria is chosen for each set.

		\subsubsection{Consideration on Borrow-Wheeler aligner}
		Borrow-Wheeler aligner sometimes clip bases from the ends of the alignment, so fragments mapped to the reverse strand are specified by their $3'$ position instead of the $5'$.
		SAM flags and CIGAR strings are needed to determine the unclipped $5'$ ends.

	\subsection{Protocol}
	Marking the duplicates can be done in two different ways:

	\begin{multicols}{2}
		\begin{itemize}
			\item MarkDUplicates from Picard, the golden standard.
			\item markdup from samtool.
				It requires the addition of mate tags to the BAM file through the fixmate command.
		\end{itemize}
	\end{multicols}

	The samtools command is faster than the Picard one as it exploits the Cigar of the mate read to correct with a simple iteration.
	However the Picard command retains more reads because samtools' command removes all the reads that have a mae mapped to a different chromosome, removing in this way structural variants.
