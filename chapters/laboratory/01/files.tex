\graphicspath{{chapters/laboratory/01/images/}}
\chapter{Relevant file's formats}

\section{FASTA format}
FASTA files contain information about a sequence and its quality score.

	\subsection{Components}
	FASTA files are composed of:

	\begin{multicols}{2}
		\begin{enumerate}
			\item A line beginning with $>$ followed by a sequence ID and a sequence description.
			\item The sequence.
			\item Quality scores for each base.
		\end{enumerate}
	\end{multicols}

	A multi-fasta file is obtained as a simple concatenation of individual FASTA, with $>$ as the separator.

	\subsection{Alphabet}
	The alphabet of FASTA is built such that for DNA and RNA:

	\begin{multicols}{2}
		\begin{itemize}
			\item $ATCG$ for the normal bases.
			\item $N$ for an unknown base.
			\item $R$ for either $A$ or $G$.
			\item $Y$ for either $C$ or $T$.
			\item For RNA $T$ is replaced by the $U$.
		\end{itemize}
	\end{multicols}

	For Proteins:

	\begin{multicols}{2}
		\begin{itemize}
			\item Standard letter for standard amino acids.
			\item $X$ for unknown amino acids.
			\item $OBZJ$ for protein amino acid modifications.
		\end{itemize}
	\end{multicols}

	\subsection{DNA sequence quality}
	DNA sequences have a quality value associated with each nucleotide.
	This score is a measure of the reliability for each base, as it is derived from the physical process of sequencing.
	The quality score is formalized by the Phred software for the human genome project: let $P$ be the probability of a base call being incorrect, then the quality score $Q$ is computed as:

	$$Q = -10\log P \Leftrightarrow P = 10^{-\frac{Q}{10}}$$

\section{FASTQ format}
FASTQ files are similar as FASTA files, their differences being that the start symbol is an $@$ instead of an $>$, and the quality score is separated from the sequence with a $+$ followed by a blank line.
Moreover the quality encoding uses letters or symbols to represent numbers.

	\subsection{Data compression}
	FASTQ files are very big, as they are typically more than $10GB$.
	Therefore they will often be compressed with \emph{gzip}, in order to get to $\le 20\%$ of their original size.

\section{SAM and BAM formats}

	\subsection{SAM files}
	SAM files contain information about on the alignment of each read, optimized for readability and sequential access.

		\subsubsection{COmposition}
		They are composed by an header containing information about:

		\begin{multicols}{2}
			\begin{itemize}
				\item Version.
				\item Reference sequences.
				\item Read groups with platform information.
				\item Processing history.
			\end{itemize}
		\end{multicols}

		Following the header alignment records are found, containing:

		\begin{multicols}{2}
			\begin{itemize}
				\item Query name.
				\item Flag.
				\item Mapping quality.
				\item CIGAR.
				\item Sequence.
				\item Quality.
			\end{itemize}
		\end{multicols}

	\subsection{BAM files}
	BAM files are binary SAM files, compressed and optimized for size.
	They may be sorted and indexed at the location query.
	A sorted and indexed BAM is the default for an analysis pipeline and it is converted into a SAM file only to allow visualization.

	\subsection{Operation with SAM and BAM files}
	Samtools by default expects a BAM file as input and will produce a SAM file as output.
	Alignment results are typically stored as a sorted and indexed BAM file.
	Aligners produce SAM files.

		\subsubsection{Converting from SAM to BAM}

		\begin{enumerate}
			\item samtools view -Sbh Normal.sam > Normal.bam
			\item samtools sort Normal.bam > Normal.sorted.bam
			\item samtools index Normal.sorted.bam
		\end{enumerate}

		The file is sorted so that read pairs are next to one another, typically with the same order as the FASTQ file.
		Sorting will depend on the next analysis method to be used.

		\subsubsection{Filtering}
		Before performing downstream analysis, the BAM file must be cleaned and processed to eliminate biases:

		\begin{enumerate}
			\item Count reads: samtools view -c Normal.sorted.bam
			\item Reads mapping to reverse strand $f$: samtools view -c -f 16 Normal.sorted.bam
			\item Reads mapping to forward strand $F$: samtools view -c -F 16 Normal.sorted.bam
			\item Mapping quality $>30$: samtools view -c -q $30$ Normal.sorted.bam
		\end{enumerate}

		\subsubsection{Explore statistics}
		Different operations are allowed to explore statistics of a bam file:

		\begin{enumerate}
			\item General statistics: samtools stats Normal.sorted.bam > Stats.txt
			\item Single base sum coverage per region: samtools bedcov CG100.bed Normal.sorted.bam > BEDCov.txt
			\item Single base depth: samtools depth -b CG100.bed Normal.sorted.bam > BEDDepth.txt
		\end{enumerate}

		\subsubsection{Mpileup}
		Mpileup allow to compute the pileup, the number of reads aligned to a reference sequence in a region:

		\begin{enumerate}
			\item samtools mpileup -r 1:3410684-3410690 Normal.sorted.bam
			\item samtools mpileup -r 1:3410684-3410690 -q 60 -Q 60 Normal.sorted.bam
		\end{enumerate}

		The different read support for a specific gene could be checked through:

		\begin{center}
			samtools bedcov CG100.bed Normal.sorted.bam | grep "TP53"
		\end{center}

		The numbers obtained from this commands should always be normalized with the total number of reads.
