\graphicspath{{chapters/laboratory/04/images/}}
\chapter{Variant annotation}

\section{Introduction}
Variant annotation is a crucial step in linking sequence variants with changes in phenotype.
Annotation results can have a strong influence on the ultimate conclusions of disease studies.
Incorrect or incomplete annotations can cause researchers both to overlook potentially disease-relevant DNA variants and to dilute interesting variants in a pool of false positives.
The annotation capabilities depend on the pipeline, the variant caller and on the filtering threshold used.


	\subsection{Annotation databases}
	Different databases can be used for annotation.

		\subsubsection{Genomic data repositories}
		Genomic data repositories include:

		\begin{multicols}{4}
			\begin{itemize}
				\item $1000$ genomes.
				\item ExAC.
				\item gnomAD: exome and genome.
				\item dbSNP.
			\end{itemize}
		\end{multicols}

		\subsubsection{Variant-disease databases}
		Variant-disease databases include:

		\begin{multicols}{3}
			\begin{itemize}
				\item ClinVar: variation and phenotype.
				\item Gencode.
				\item HGMD: mutations.
				\item COSMIC: human cancer.
				\item OMIM: mendelian.
			\end{itemize}
		\end{multicols}

		\subsubsection{Conservation and pathogenicity prediction databases}
		Conservation and pathogenicity prediction databases provide computational methods for prediction and include:

		\begin{multicols}{3}
			\begin{itemize}
				\item SIFT: impact of missense variants.
				\item PolyPhen2: deleteriousness of change.
				\item GERP$++$: sequence conservation.
			\end{itemize}
		\end{multicols}

\section{SnpEff}

	\subsection{Introduction}
	SnpEff is a variant effect predictor program categorizing each variant based on its relationship to coding sequenced in the genome and how it may change the coding sequence and affect the gene product.

	\subsection{Set of transcripts}
	Variant annotation depends on the set of transcripts used as the bases for annotation.
	Widely used annotation databases such as ENSEMBL, RefSeq and UCSC contain sets of transcripts that can be used for variant annotation.

	\subsection{Populating the VCF}
	SnpEff take information from the provided annotation database and populate a VCF file by adding it itno the INFO field name ANN.
	Data fields are encoded separated by the pipe sign $|$ and the order of fields is written in the VCF header.
	More than $4GB$ are necessary to run a full annotation.
	Depending on the database a more or less complete annotation will be obtained.

	\subsection{Common annotations}
	Some common annotations include:

	\begin{multicols}{2}
		\begin{itemize}
			\item Putative\_impact or impact: a simple estimation of putative impact or deleteriousness, can take values HIGH, MODERATE, LOW, MODIFIER.
			\item Gene name: common gene name HGNC.
				Optionally it can use the closest gene when the variant is intergenic.
			\item Feature type: with type of feature, for example, transcript, motif or miRNA.
				It is preferred to use Sequence Ontology SO terms, but custom ones are allowed.
			\item Feature ID: this may be the transcript ID, the motif ID, the miRNA, ChipSeq peak or histone mark for example.
				Some features may not have a unique ID.
			\item Byotipe: a description on whether the transcript is coding or non-coding.
		\end{itemize}
	\end{multicols}

	\subsection{SnpSift}
	Each gene can be present in many transcript evrsions, so the estimated impact is different.
	SnpSift helps filter large genomic datasets in order to find the most significant variants.
	Complex expression to filter and combine can be used while annotation.
	All fields in ANN can be managed.
