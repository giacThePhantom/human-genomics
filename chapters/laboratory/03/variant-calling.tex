\graphicspath{{chapters/laboratory/03/images/}}
\chapter{Variant calling}

\section{Introduction}

	\subsection{Objective of variant calling}
	The objective of variant calling is to compute allelic fractions across all positions.
	Thresholds or basic statistics can be used to distinguish between genotypes.
	The goal is to identify germline variants.
	Quality thresholds can be added.

	\subsection{Bayes' rule for variant calling}
	Refined methods can find the genotype of each sample by calculating via Bayes' rule the probability of each possible genotype/
	Let $P(G)$ be the prior of the genotype, $P(D|G)$ the likelihood of the genotype and $P(D|H)$ the haploid likelihood function.
	Then, given $G = H_1H_2$ the diploid assumption, the Bayesian model is computed as:

	$$\begin{cases}P(G|D) = \frac{P(G)P(D|G)}{\sum\limits_iP(G_i)P(D|G_i)}\\P(D|G) = \prod\limits_j\big(\frac{P(D_j|H_1)}{2} + \frac{P(D_j|H_2)}{2}\big)\end{cases}$$

	Where:

	$$P(D_j|H) = P(D_j|b)$$

	Is a single base pileup and:

	$$P(D_j|b) = \begin{cases}1-\epsilon_j & D_j = b\\\epsilon_j & D_j\neq b\end{cases}$$


\section{Likelihood estimation for variant calling}
The inference used as a gold standard relies on a likelihood function to estimate the probability of sample data given the proposed haplotype.
The probability is computed by calculating the support of the alternative base based on quality.
All diploid genotypes are considered at each base.
The likelihood of genotype is computed using only pileup of bases and associated quality scores at a given locus.
Only bases satisfying minimum base quality, mapping read quality, pair mapping quality or other quality parameter are considered.
This approach can be implemented as a multi-sample calling, where it is possible to obtain joint estimates across samples.

	\subsection{Available tools}
	Available tools to perform this task are:

	\begin{itemize}
		\item Bcftools: bcftools mpileup -Ou -a DP -f human\_g1k\_v37.fasta Sample.sorted.bam | bcftools call -Ov -c -v $>$ Sample.BCF.vcf
		\item GATK: java -jar GATK/GenomeAnalysisTK.jar -T UnifiedGenotyper -R human\_g1k\_v37.fasta -I Sample.sorted.bam -o Sample.GATK.vcf -L chr20.bed
	\end{itemize}

\section{VCF files}
VCF is a text file format, most likely stored in a compressed matter, that contains meta-information lines, a header line and data lines, each containing information about a position in the genome.

	\subsection{Composition}
	The fileformat field is required and it should detail the VCF format version number.
	Descriptions of field INFO, FILTER and FORMAT are optional but strongly encouraged.
	The header line names $8$ fixed, mandatory columns:

	\begin{multicols}{2}
		\begin{itemize}
			\item CHROM: chromosome.
			\item POS: position.
			\item ID: semi-colon separated list of unique identifiers.
			\item REF: reference bases.
			\item ALT: comma separated list of alternative non-reference alleles called on at least one of the samples.
			\item QUAL: prhed-scaled quality score for the assertion made in ALT.
			\item FILTER: pASS if the position has passed all filters.
				"." when filters are not applied.
			\item INFO: additional information.
		\end{itemize}
	\end{multicols}

	Genotype data are followed by a FORMAT column header and an arbitrary number of sample IDs.
	The header line is tab-delimited.

	\subsection{Vcftools}
	The package vcftools can be used to perform a number of operations on VCF files:

	\begin{multicols}{2}
		\begin{itemize}
			\item Filter out specific variants for quality, mean depth or allelic fraction.
			\item Compare files.
			\item Summarize variants.
			\item Convert to different file types.
			\item Validate and merge files.
			\item Create intersections and subsets of variants.
		\end{itemize}
	\end{multicols}

\section{A variant calling protocol}
A variant calling protocol consists of two steps.
In the first bcftools pileup provides the supported bases and their quality.
The highest support is checked through a Bayesian model.
A position with an alternative allele is reported in the sample VCF file.
Then in the second step vcftools or GATK are used to produce a new VCF with the corresponding informations.
GATK keeps more variants with respect to vcftools, but there is no method clearly better than the other.
Usually the consensus is used to compare the output of the two tools and an intersection is built.
Moreover the two tools show a lot of similarities in their output.
Vcftools are more confident in making calls while GATK can introduce a portion of false positives.
This can be reduced by increasing the filter on minimal coverage and reaching a set of more trustable parameters.
