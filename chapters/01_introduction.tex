\chapter{Introduction}

\section{Definitions}

	\subsection{Genetics}
	Genetics is the study of heredity, or how the characteristics of living organisms are transmitted from one generation to the next via DNA.
	It dates back to Augustinian friar and scientist Gregor Mendel.
	It involves the study of a specific and limited number of genes or their part that have a known function.

	\subsection{Genomics}
	Genomics is the study of the entirety of an organism's genes, the genome.
	Using high-performance computing and moth techniques known as bioinformatics, genomics researchers analyse enormous amounts of DNA-sequence data to find variations that affect health, disease or drug response.
	In human that means searching through about $3$ billion units of DNA across $23000$ genes.

	\subsection{Differences}
	The main difference between genomics and genetics is that genetics scrutinizes the functioning and composition of the single gene, where genomics addressees all genes and their relationships in order to identify their combined influence on the growth and development of the organism.

	\subsection{Role of computational biology}
	Computational biology offer a wide range of numerical methods to analyse and integrate large scale data towards the understanding of molecular, cellular and structural biology.
	The focus of this course is on human genomics and how to mine raw data, how to exploit it for quality control and how to interpret the results in the context of human disease, especially cancer.

\section{Differences in genetic make-up}
The genetic make-up of individual is different between individuals and it is responsible for human diversity.
These variants can be inherited or acquired.

	\subsection{Inherited variants}
	Between inherited variants single nucleotide polymorphisms and copy number variants can be identified.

		\subsubsection{Single nucleotide polymorphisms}
		Single nucleotide polymorphisms or SNPs are changes of one nucleotide in the sequence of a gene.
		They constitute $1\%$ of the difference between two unrelated individuals' genomes.

		\subsubsection{Copy number variants}
		Copy number variants or CNVs are difference of the number of allele for a gene present in one individual.
		They contribute much more than SNPs in the difference between unrelated individuals.

		\subsubsection{Characteristics of inherited variants}
		Inherited variants can be characterized by penetrance and allele frequency.

			\paragraph{Penetrance}
			Penetrance is the proportion of individuals carrying an allele or genotype that also expresses the trait or phenotype associated with it.

			\paragraph{Allele frequency}
			Allele frequency is the ratio between the number of times the allele of interest is observed in a population over the total number of copies of all the alleles at that particular genetic locus in the population.

	\subsection{Acquired DNA aberrations}
	DNA aberrations happen in diseased or aged cells and are the key to cancer genomics.
	They are also called somatic variants: they are not inherited from parents and are not transmitted to offspring.
	They can be single nucleotide variants or SNV or point mutation, indels or deletions, rearrangements and somatic copy number aberrations or SCNA.

		\subsubsection{Types of acquired DNA aberrations}

			\paragraph{Translocation}
			Translocation happens when a sequence is moved from one genetic locus to another.
			It can be an insertion or unbalanced, where only one sequence move or balanced, when two sequences exchange locus.

			\paragraph{Inversion}
			Inversion happens when a sequence inverts its orientation.

			\paragraph{Duplication}
			In duplication a sequence doubles its copy number.

			\paragraph{Deletion}
			In deletion a sequence is lost.

			\paragraph{Chromoplexy}
			Chromoplexy is a class of complex somatic DNA rearrangements whereby abundant DNA deletions and intra and inter-chromosomal translocations that have originated in an interdependent way occur in a single cell cycle.

			\paragraph{Chromothripsis}
			Chromotripsis is a clustered chromosomal rearrangement in confined genomic regions that result from a single catastrophic event, usually limited to one chromosome.

			\paragraph{Kataegis}
			Kataegis is a phenomenon that is characterized by large clusters of mutations in the genome of cancer cells.
			An APOBEC family enzyme might be responsible for the kataegis process.
